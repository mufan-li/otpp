% Borrowed from MIT 6.987 Advanced Data Structures course (Prof Demaine, 2003)

\documentclass[11pt]{article}
\usepackage[utf8]{inputenc} 
\usepackage{amsmath, amssymb, amsthm, bm}
\usepackage[usenames,dvipsnames]{xcolor}
\usepackage[colorlinks,citecolor=blue,urlcolor=blue,linkcolor=blue]{hyperref}
\usepackage{hypernat}

\usepackage{tikz}
\usepackage{tkz-graph}

\usepackage{listings}             % Include the listings-package
\usepackage{algorithm2e}
\usepackage{dsfont}

\newcommand{\handout}[6]{
  \noindent
  \begin{center}
  \framebox{
    \vbox{
      \hbox to 5.78in { {\bf } \hfill #2 }
      \vspace{4mm}
      \hbox to 5.78in { {\Large \hfill #5  \hfill} }
      % \vspace{2mm}
      \hbox to 5.78in { {\em #3 \hfill #4} }
      \hbox to 5.78in { {\em \hfill #6} }
    }
  }
  \end{center}
  \vspace*{4mm}
}

\newcommand{\lecture}[4]{\handout{#1}{#2}{#3}{#4}{Global Credit Products - Case Study Report #1}}

\newtheorem{theorem}{Theorem}
\newtheorem{corollary}[theorem]{Corollary}
\newtheorem{lemma}[theorem]{Lemma}
\newtheorem{observation}[theorem]{Observation}
\newtheorem{proposition}[theorem]{Proposition}
\newtheorem{definition}[theorem]{Definition}
\newtheorem{claim}[theorem]{Claim}
\newtheorem{fact}[theorem]{Fact}
\newtheorem{assumption}[theorem]{Assumption}
\newtheorem{remark}[theorem]{Remark}

\theoremstyle{definition}
\newtheorem{problem}[theorem]{Problem}

\makeatletter
\newcommand{\xRightarrow}[2][]{\ext@arrow 0359\Rightarrowfill@{#1}{#2}}
\makeatother

% 1-inch margins, from fullpage.sty by H.Partl, Version 2, Dec. 15, 1988.
\topmargin 0pt
\advance \topmargin by -\headheight
\advance \topmargin by -\headsep
\textheight 8.9in
\oddsidemargin 0pt
\evensidemargin \oddsidemargin
\marginparwidth 0.5in
\textwidth 6.5in

\parindent 0in
\parskip 1.5ex
%\renewcommand{\baselinestretch}{1.25}

\begin{document}

\lecture{}{May 22, 2016}{}{Mufan Li}{mufan.li@mail.utoronto.ca}

\section{Convertible Bond Pricing}

This section describes the solution to 
the first part of the case study on pricing of 
a convertible bond.
Please refer to the attached OTPP\_Convertible\_Bond.xlsm file 
for the computation in Excel VBA. 
The specific details will be explain in Section 1.3.

\subsection{Description of a Convertible Bond}

A \emph{convertible bond} is a derivative security
that depends on the underlying stock price.
Specifically, the convertible bond behaves as a regular
bond while also providing the holder a right to convert 
the bond into stock at a prespecified ratio.
For this problem, our convertible bond will continue 
to pay a semiannual coupon until maturity without 
exercise of the conversion (and default);
when exercised, each bond will convert into 
a number (\emph{conversion ratio}) of stock.

Since exercise is allowed at anytime, 
the security behaves like an American option:
it will always take the maximum of the holding value and
the exercise value (also known as \emph{parity price}).
Due to this complexity, 
there does not exist a closed-form solution 
for the price, and we must rely on a numerical 
approximation instead.

It is also common for the convertible bond to be 
callable or puttable;
a \emph{callable bond} allows the issuer of the bond
to buy back the bonds before the maturity date;
similarly a \emph{puttable bond} allows the holder
of the bond to demand early repayment of bond.
In this problem, we will not deal with the complexity of 
such instruments.





\subsection{The Binomial Tree Approach to Pricing}

A \emph{binomial tree} (also known as 
\emph{binomial lattice}) is a discretized model of 
a continuous stochastic process,
such that at any stage, the process can only evolve 
to two potential values in the next step.
As the size of the time steps is taken to zero,
the discretized process will converge to 
the continuous process (under some technical conditions).
In practice, the convergence is more than sufficient 
for modeling simple derivative prices.

In this problem, we allow a third potential value 
in the event of a default, where the stock price 
will become zero, and the bond price will be equal to 
the \emph{recovery value}.
The underlying principle remains the same
for find the price of the convertible bond $P_\text{cb}$
%
\begin{equation*}
  P_\text{cb}(t,S(t)) = \max\Big\{P_\text{exercise}(t,S(t)), 
                        P_\text{hold}(t,S(t)) + 
                        P_\text{coupon}(t)\Big\}
\end{equation*}
%
where $P_\text{coupon}(t)$ is the coupon payment,
and $P_\text{exercise}(t,S(t))$ is the value of the conversion
at conversion rate $C_r$
%
\begin{equation*}
  P_\text{exercise}(t,S(t)) = C_r S(t)
\end{equation*}
%
$P_\text{hold}(t,S(t))$ can be computed as the 
discounted (conditional) expectation of 
the price at the next time step
%
\begin{equation*}
\begin{aligned}
  P_\text{hold}(t,S(t)) &= e^{-r\Delta t} 
    \mathbb{E}\Big[P_\text{cb}(t+\Delta t,S(t+\Delta t)) \Big| S(t)\Big] \\
  &= e^{-r\Delta t} \Big[p_\text{up} 
    P_\text{cb}(t+\Delta t,S_\text{up}(t+\Delta t)) + 
    p_\text{down} P_\text{cb}(t+\Delta t,S_\text{down}(t+\Delta t)) + 
    p_\text{default} P_\text{recovery}]
\end{aligned}
\end{equation*}
%
Here $p_\text{up}, p_\text{down}, p_\text{default}$
are (risk-neutral) probabilities of each of the branches
in the binomial tree, $r$ is the risk-free rate,
and $S_\text{up}, S_\text{down}$ correspond to 
the up and down branches of the binomial tree of the stock.

Observe that $P_\text{cb}(t,S(t))$ is can be found recursively
from the values of $P_\text{cb}(t+\Delta t,S(t+\Delta t))$.
Therefore, by determining the values of $P_{cb}(T)$ 
where $T$ is the maturity time.
Since the holding value of convertible bond at maturity
is just the par value plus the final coupon, we have
%
\begin{equation*}
  P_\text{cb}(T) = \max\Big\{C_r S(T), P_\text{par} + 
                    P_\text{coupon}(T)\Big\}
\end{equation*}


\subsection{VBA Implementation}

\subsection{Computing Delta}


































\end{document}
