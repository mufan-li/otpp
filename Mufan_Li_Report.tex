% Borrowed from MIT 6.987 Advanced Data Structures course (Prof Demaine, 2003)

\documentclass[11pt]{article}
\usepackage[utf8]{inputenc} 
\usepackage{amsmath, amssymb, amsthm, bm}
\usepackage[usenames,dvipsnames]{xcolor}
\usepackage[colorlinks,citecolor=blue,urlcolor=blue,linkcolor=blue]{hyperref}
\usepackage{hypernat}

\usepackage{tikz}
\usetikzlibrary{matrix}
\usepackage{tkz-graph}
\usepackage{listings}             % Include the listings-package
\usepackage{algorithm2e}
\usepackage{dsfont}
\usepackage{epsfig}

\newcommand{\handout}[6]{
  \noindent
  \begin{center}
  \framebox{
    \vbox{
      \hbox to 5.78in { {\bf } \hfill #2 }
      \vspace{4mm}
      \hbox to 5.78in { {\Large \hfill #5  \hfill} }
      % \vspace{2mm}
      \hbox to 5.78in { {\em #3 \hfill #4} }
      \hbox to 5.78in { {\em \hfill #6} }
    }
  }
  \end{center}
  \vspace*{4mm}
}

\newcommand{\lecture}[4]{\handout{#1}{#2}{#3}{#4}{Global Credit Products - Case Study Report #1}}

\newtheorem{theorem}{Theorem}
\newtheorem{corollary}[theorem]{Corollary}
\newtheorem{lemma}[theorem]{Lemma}
\newtheorem{observation}[theorem]{Observation}
\newtheorem{proposition}[theorem]{Proposition}
\newtheorem{definition}[theorem]{Definition}
\newtheorem{claim}[theorem]{Claim}
\newtheorem{fact}[theorem]{Fact}
\newtheorem{assumption}[theorem]{Assumption}
\newtheorem{remark}[theorem]{Remark}

\theoremstyle{definition}
\newtheorem{problem}[theorem]{Problem}

\makeatletter
\newcommand{\xRightarrow}[2][]{\ext@arrow 0359\Rightarrowfill@{#1}{#2}}
\makeatother
\DeclareMathOperator*{\argmax}{arg\,max}
\DeclareMathOperator*{\argmin}{arg\,min}

% 1-inch margins, from fullpage.sty by H.Partl, Version 2, Dec. 15, 1988.
\topmargin 0pt
\advance \topmargin by -\headheight
\advance \topmargin by -\headsep
\textheight 8.9in
\oddsidemargin 0pt
\evensidemargin \oddsidemargin
\marginparwidth 0.5in
\textwidth 6.5in

% \parindent 0in
% \setlength{\parindent}{1cm} % Default is 15pt.
\parskip 1.5ex
%\renewcommand{\baselinestretch}{1.25}

\lstset{
  captionpos=b,   
}

\begin{document}

\lecture{}{May 22, 2016}{}{Mufan Li}{mufan.li@mail.utoronto.ca}

\section{Convertible Bond Pricing}

This section describes the solution to 
the first part of the case study on pricing of 
a convertible bond.
Please refer to the attached OTPP\_Convertible\_Bond.xlsm file 
for the computation in Excel VBA. 
The specific details will be explain in Section \ref{ssc:vba}.

\subsection{Description of a Convertible Bond}

A \emph{convertible bond} is a derivative security
that depends on the underlying stock price.
Specifically, the convertible bond behaves as a regular
bond while also providing the holder a right to convert 
the bond into stock at a prespecified ratio.
For this problem, our convertible bond will continue 
to pay a semiannual coupon until maturity without 
exercise of the conversion (and default);
when exercised, each bond will convert into 
a number (\emph{conversion ratio}) of stock.

Since exercise is allowed at anytime, 
the security behaves like an American option:
it will always take the maximum of the holding value and
the exercise value (also known as \emph{parity price}).
Due to this complexity, 
there does not exist a closed-form solution 
for the price, and we must rely on a numerical 
approximation instead.

It is also common for the convertible bond to be 
callable or puttable;
a \emph{callable bond} allows the issuer of the bond
to buy back the bonds before the maturity date;
similarly a \emph{puttable bond} allows the holder
of the bond to demand early repayment of bond.
In this problem, we will not deal with the complexity of 
such instruments.





\subsection{The Binomial Tree Approach to Pricing}

A \emph{binomial tree} (also known as 
\emph{binomial lattice}) is a discretized model of 
a continuous stochastic process,
such that at any stage, the process can only evolve 
to two potential values in the next step.
As the size of the time steps is taken to zero,
the discretized process will converge to 
the continuous process (under some technical conditions).
In practice, the convergence is more than sufficient 
for modeling simple derivative prices.

In this problem, we allow a third potential value 
in the event of a default, where the stock price 
will become zero, and the bond price will be equal to 
the \emph{recovery value}.
The underlying principle remains the same
for find the price of the convertible bond $P_\text{cb}$
%
\begin{equation}
\label{exercise}
  P_\text{cb}(t,S(t)) = \max\Big\{P_\text{exercise}(t,S(t)), 
                        P_\text{hold}(t,S(t)) + 
                        P_\text{coupon}(t)\Big\}
\end{equation}
%
where $P_\text{coupon}(t)$ is the coupon payment function
(point masses at coupon payment dates),
and $P_\text{exercise}(t,S(t))$ is the value of the conversion
at conversion rate $C_r$
%
\begin{equation*}
  P_\text{exercise}(t,S(t)) = C_r S(t)
\end{equation*}
%
$P_\text{hold}(t,S(t))$ can be computed as the 
discounted (conditional) expectation of 
the price at the next time step
%
\begin{equation}
\label{cond_exp}
\begin{aligned}
  P_\text{hold}(t,S(t)) &= e^{-r\Delta t} 
    \mathbb{E}\Big[P_\text{cb}(t+\Delta t,S(t+\Delta t)) \Big| S(t)\Big] \\
  &= e^{-r\Delta t} \Big[p_\text{up} 
    P_\text{cb}(t+\Delta t,S_\text{up}(t+\Delta t)) + 
    p_\text{down} P_\text{cb}(t+\Delta t,S_\text{down}(t+\Delta t)) + 
    p_\text{default} P_\text{recovery} \Big]
\end{aligned}
\end{equation}
%
Here $p_\text{up}, p_\text{down}, p_\text{default}$
are (risk-neutral) probabilities of each of the branches
in the binomial tree, $r$ is the risk-free rate,
and $S_\text{up}, S_\text{down}$ correspond to 
the up and down branches of the binomial tree of the stock.

\begin{figure}[h]
\centering
\begin{tikzpicture}[>=stealth,sloped]
    \matrix (tree) [%
      matrix of nodes,
      minimum size=1cm,
      column sep=3.5cm,
      row sep=1cm,
    ]
    {
          &    \\
          & $P_\text{cb}(t+\Delta t, S_\text{up}(t+\Delta t))$  \\
      $P_\text{hold}(t,S(t))$  &    \\
          & $P_\text{cb}(t+\Delta t, S_\text{down}(t+\Delta t))$  \\
          &     \\
          & $P_\text{recovery}$  \\
    };
    \draw[->] (tree-3-1) -- (tree-2-2) node [midway,above] {$p_\text{up}$};
    \draw[->] (tree-3-1) -- (tree-4-2) node [midway,below] {$p_\text{down}$};
    \draw[->] (tree-3-1) -- (tree-6-2) node [midway,below] {$p_\text{default}$};
\end{tikzpicture}
\caption{One step binomial tree for to find the value of 
  holding the convertible bond,
  corresponding to Equation \eqref{cond_exp}.}
\end{figure}


Observe that $P_\text{cb}(t,S(t))$ is can be found recursively
from the values of $P_\text{cb}(t+\Delta t,S(t+\Delta t))$.
Therefore, by determining the values of $P_{cb}(T)$ 
where $T$ is the maturity time.
Since the holding value of convertible bond at maturity
is just the par value plus the final coupon, we have
%
\begin{equation}
\label{terminal}
  P_\text{cb}(T,S(T)) = \max\Big\{C_r S(T), P_\text{par} + 
                    P_\text{coupon}(T)\Big\}
\end{equation}
%
With this terminal condition, we can recursively find
$P_\text{cb}(T-\Delta t), P_\text{cb}(T- 2 \Delta t),
\ldots, P_\text{cb}(0)$,
where the present value of the bond is found at $t=0$.








\subsection{VBA Implementation} \label{ssc:vba}

In the VBA code, we will work with two arrays \emph{ConvertNext}
and \emph{ConvertCurrent}, 
corresponding to the values of $P_\text{cb}(t+\Delta t)$
and $P_\text{cb}(t)$.
And as we step backward in time, the two arrays will be resized as needed.

We start by applying the terminal condition in Equation \eqref{terminal},
which is implemented by the following code

\begin{lstlisting}[language = VBScript, caption = VBA Code for Terminal Condition]

For State = 0 To N_steps
    stock = S0 * (u ^ State) * (d ^ (N_steps - State))

    ConvertNext(State) = Application.Max(conv_ratio * stock, _
                                        par + coupon_val)
Next State

\end{lstlisting}

After which, we can then apply the conditional expectation 
in Equation \eqref{cond_exp} 
to assign the value of variable \emph{Convert}.
Finally \emph{ConvertCurrent} is determined
by applying the exercise constraint in Equation \ref{exercise}.

\begin{lstlisting}[language = VBScript, caption = VBA Code for Terminal Condition]

For State = 0 To Index
    stock = S0 * (u ^ State) * (d ^ (Index - State))

    Convert = Pd * ConvertNext(State) + Pu * ConvertNext(State + 1) _
                + (1 - Exp(-lambda * dt)) * Recovery * par

    ConvertCurrent(State) = Application.Max(conv_ratio * stock, _
                            Exp(-R * dt) * Convert + coupon)
Next State

\end{lstlisting}

To finish the iteration, we copy the values in 
\emph{ConvertCurrent} to \emph{ConvertNext} 
before the next loop.

After exiting the loop,
we will finally end up with only one node in the 
binomial tree,
which is the value of \emph{ConvertCurrent(1,1)}
corresponding to $P_\text{cb}(0)$.








\subsection{Computing Delta}

The \emph{delta} of the convertible bond is 
%
\begin{equation*}
  \Delta = \left. \frac{\partial P_\text{cb}(t,S(t)) }{\partial S(t)}
  \right|_{t=0}
\end{equation*}

For this part of the problem, we consider two approaches. 
First by centered finite difference approximation, 
we know that
%
\begin{equation} \label{cfd}
  \frac{\partial P_\text{cb}(t,S(t)) }{\partial S(t)}
  = \frac{P_\text{cb}(t,S(t) + \epsilon) - 
          P_\text{cb}(t,S(t)-\epsilon)}{2\epsilon}
    + \mathcal{O}(\epsilon ^ 2)
\end{equation}
%
where $\mathcal{O}(\cdot)$ refers to the order of 
the discretization error.
With this method, we have a 
convergence guarantee by Taylor's Theorem;
however, this requires two additional computations 
of the binomial trees.

Since the binomial tree is already computed once,
we can then use the first branches to approximate the delta as well
to save computations. Here we use the following approximation
%
\begin{equation} \label{bino_cfd}
  \frac{\partial P_\text{cb}(t,S(t)) }{\partial S(t)}
  = \frac{P_\text{cb}(t + \Delta t, u S(t) ) - 
        P_\text{cb}(t + \Delta t, d S(t)-\epsilon)}{
        (u-d)S(t)}
    + \mathcal{O}\left(\Delta t + [(u-d)S(t)]^2\right)
\end{equation}
%
where $u,d$ are the returns of the up and down branches.

There are two issues with this approximation.
Firstly the error is capped by the discretization size in time
instead of a user chosen $\epsilon$.
Although this can be resolved by choosing a very small 
$\Delta t$ in the beginning to improve the approximation,
it will require a different implementation.
Secondly, this approximation ignores the third branch 
in the event of a default.
Therefore, we expect this estimate to be worse
while not repeating computation.

When comparing the two estimates for this problem, 
the difference is quite small:
%
\begin{equation*}
\begin{aligned}
  \text{Finite Difference: } 1.8432617 \\
  \text{Binomial Branches: } 1.8411774
\end{aligned}
\end{equation*}
%
where we chose $\epsilon = 10^{-6}$. 
Therefore we can conclude that for this problem, 
it is sufficient to approximate by 
using the first branches in the binomial tree.















\pagebreak
\section{Trend Following Model}

This part of the report investigates the trend following models.
The script file that drives combines all the functions is in
\emph{trend.m}, while all the other \emph{.m} defines 
the helper functions as required.
Note here \emph{helper.m} implements a placeholder class
such that all the extra helper functions can be 
combined in one \emph{.m} file.

\vspace{5mm}
\noindent
\textbf{Model Assumptions (Interpretation)}

\begin{itemize}
  \item When computing the CDS spread for 5 year maturity,
    it is assumed that the CDS spread desired is 
    the CDS contract with maturity of exactly 5 years.
    This value is linearly interpolated between
    the 5 year standard contract (which always has a maturity 
    of longer than 5 years), and 4 year standard contract 
    (which has a maturity of less than 5 years).
  \item When computing the returns of each CDS series,
    it is assumed that \emph{cv01(s)} can be \emph{extrapolated} 
    when the spread is greater than 5000,
    as the CDS series 59 does at one point reach a 
    spread of greater than 10000.
  \item When implementing the trading rule,
    it is assumed that ``sell protection 1 million CDS notional'' 
    refers to taking a short position (long risk) in the CDS, 
    instead of adding an additional 1 million short position 
    into the portfolio. 
    Otherwise, this will potentially lead to 
    trading every CDS on every day.
\end{itemize}

\subsection{The Basic Model and Summary Statistics}

Here we look at several summary statistics
of the basic strategy.
Return is calculated as profit in dollars as a percent of 
notional value (\$ 1 million), 
assuming we do not re-invest the profit,
which results in the cumulative profit 
computed as a sum of dollar profits 
instead of a cumulative product of returns.
This assumption is used since returns in CDS 
are in the form of a cash flow,
however it will have minimal influence on the analysis.
The Sharpe ratio calculation is computed based on
daily returns annualized and using zero risk-free rate.
The kurtosis refers to the non-excess kurtosis,
and both skewness kurtosis are computed for daily 
portfolio returns.

\begin{table}[h]
\centering
\begin{tabular}{ | l | r | }
  \hline      
  Statistic & Value \\
  \hline
  Cumulative Return & 7.01\% \\
  Annualized Return & 2.87\% \\
  Sharpe Ratio & 4.0256 \\
  Skewness & -0.2167 \\
  Kurtosis & 14.4875 \\
  \hline  
\end{tabular}
\caption{Summary Statistics of the basic strategy.}
\label{tab:basic_summary}
\end{table}

Based on these statistics alone, especially the Sharpe ratio, 
this strategy does look attractive. 
However, the kurtosis is relatively large
even when compared to a t-distribution with 5 degrees of freedom
(kurtosis of 9),
indicating a very heavy tail.

\subsection{Number of Portfolio Components}

The most basic model in the previous part 
used all 343 CDS series to 
compute the return, when it is potentially unnecessary.
While clearly diversifying can reduce volatility 
of the portfolio, which improves the Sharpe ratio,
there should be a point at which the improvement 
will begin diminishing.
We can investigate the performance of portfolio 
of different sizes and seek an approximate 
optimal size.
Since we do not have external information to 
select the best CDS series,
we can look at random samples from the 343 series 
and compute their return statistics.

Specifically we randomly choose $N_\text{sample}$ out of 
$N = 343$, a total of 50 times,
and compute the summary statistics for all 50 scenarios.
We can use the box plot to observe the change in 
distributions for the annualized return and Sharpe ratio
as we vary the value of $N_\text{sample}$.

\begin{figure}[h]
\begin{center}
\epsfig{figure=boxplot_sharpe.eps, width=.51\textwidth}
\end{center}
\centering
\caption{\label{fg:boxplot_sharpe}
The box plot of Sharpe ratio for sampled portfolio
over different sample sizes.
}
\end{figure}

\begin{figure}[h]
\begin{center}
\epsfig{figure=boxplot_return.eps, width=.51\textwidth}
\end{center}
\centering
\caption{\label{fg:boxplot_return}
The box plot of annualized return for sampled portfolio
over different sample sizes.
}
\end{figure}

From the box plots, we can observe the improvement 
in average Sharpe ratio begins to diminish 
after reaching a sample size of 50.
Therefore, we can conclude that it is unnecessary 
to trade all 343 CDS series, 
instead 50-100 components will be able to reach similar results.

\subsection{Missing Assumptions and Transaction Cost}

A key difference between back-testing and 
actual trading is the execution cost
due to available liquidity, trade size, 
and broker commission.
Since there is no additional information on 
individual CDS series and date,
it is appropriate to assume a uniform 
transaction cost on each trade.

Here we assume the transaction cost is 
directly proportional to the notional value 
of the trade, 
which implies that each time a portfolio 
position flips from long \$1 million to short 
\$1 million, we have traded \$2 million 
of notional value.
After several guesses, it is easily found that 
the break-even transaction cost is approximately
5.41 basis points.
We can look at the summary statistics again:

\begin{table}[h]
\centering
\begin{tabular}{ | l | r | }
  \hline      
  Statistic & Value \\
  \hline
  Cumulative Return & 0.00\% \\
  Annualized Return & 0.00\% \\
  Sharpe Ratio & -0.0010 \\
  Skewness & -0.1464 \\
  Kurtosis & 14.1028 \\
  \hline  
\end{tabular}
\caption{Summary Statistics of the basic strategy
  with transaction cost of 5.41 basis points.}
\label{tab:summary2}
\end{table}

Clearly, a transaction cost of 5.41 basis points 
is very small and unrealistic.
Therefore the strategy, in its current state,
is not very attractive.

\subsection{Strategy's Potential}

If 5.41 basis points of transaction cost
is enough to break even an annualized returns of 2.87\%, 
clearly direct application of the strategy 
induces a high trading volume.
Perhaps some of the trading activity is unnecessary.
We can get a rough idea of trading activity 
by simply looking at the sparsity of 
the matrix of trading volume.

\begin{figure}[h!]
\begin{center}
\epsfig{figure=spy_volume.eps, width=.51\textwidth}
\end{center}
\centering
\caption{\label{fg:spy_volume}
A (sample) sparsity plot of the trading volume matrix,
where each point indicates a non-zero entry
in the matrix, i.e. a trade.
}
\end{figure}

Here we observe Figure \ref{fg:spy_volume} 
that the strategy tend to 
frequently reverse its position on a single CDS series,
as often as 4 days in a row.
This type of trading is due to the moving average 
oscillating between positive and negative values, 
causing the strategy to trade frequently.
Therefore, the strategy has potential to 
significantly reduce its transaction costs
by eliminating this type of trades.

Additionally, the choice of
using a 10-day moving average is arbitrary.
This length of moving average 
can be optimized for better results.

\subsection{Optimized Strategy}

As mentioned in the previous section, 
the strategy can be improved by both reducing 
unnecessary trading volume, 
and by optimizing the length of simple moving average.

Here we investigate a simple modification to 
the trading rule by introducing a tolerance to 
the moving average,
where we will continue to hold the current position 
until the moving average has crossed to 
the opposite sign by at least some 
$\epsilon_\text{tol} > 0$.
More formally
%
\begin{equation*}
  \text{Position on CDS}_i =
  \begin{cases}
    -1 & \text{ if MA}_t(S_i,T) > \epsilon_\text{tol} \\
    +1 & \text{ if MA}_t(S_i,T) < -\epsilon_\text{tol} \\
    \text{Hold previous position} & \text{ otherwise}
  \end{cases}
\end{equation*}
%
where $T$ is the length of the moving average.

Here we can formulate the strategy choice as 
an optimization problem over 
the parameters $\{T, \epsilon_\text{tol}\}$,
with the objective to maximize some combination 
of returns and Sharpe ratio.
Since the choice of $T$ has very little impact 
over small changes, and minimization over 
one variable is a much simpler problem,
we decide to pursue the following formulation:
%
\begin{equation} \label{eq:optimization}
  \epsilon_\text{tol}^*(T) = 
  \argmax_{\epsilon_\text{tol} \in [0, 0.01]}
    100 r_\text{annual}(T,\epsilon_\text{tol}) + 
    R_\text{Sharpe}(T,\epsilon_\text{tol})
\end{equation}
%
where the optimization problem will be solved 
over several choices of $T$.
Here $r_\text{annual}$ is the annual return,
$R_\text{Sharpe}$ is the annualized Sharpe ratio,
and the factor $100$ is chosen so the two values 
are on the same order of magnitude.

\begin{table}[h!]
\centering
\begin{tabular}{ | l | r | r | r | r | r | r | r | r | }
  \hline      
  MA Length $(T)$ & 3 & 5 & 7 & 10 & 15 & 20 & 30 & 50 \\
  Tolerance $(\epsilon_\text{tol})$ &
    0.0024 & 0.0040 & 0.0061 & 0.0043 & 0.0068 & 0.0062 & 0.0034 & 0.0038 \\
  \hline
  Cumulative Return & 
    1.47\% & 1.14\% & 0.19\% & 0.54\% & -0.01\% & 0.06\% & -0.14\% & -0.16\% \\
  Annualized Return & 
    \textbf{1.21\%} & 0.93\% & 0.15\% & 0.44\% & -0.01\% & 0.05\% & 
    -0.12\% & -0.13\% \\
  Sharpe Ratio & 
    2.1592 & \textbf{2.3534} & 0.5359 & 1.3905 & -0.0378 & 0.2265 & 
    -0.4462 & -0.7317 \\
  Skewness & 
    0.4409 & 1.2342 & 0.4260 & 0.7464 & -0.3771 & -0.2468 & -0.4187 & -0.8695 \\
  Kurtosis & 
    7.1357 & 9.4457 & 6.6211 & 6.8286 & 7.3626 & 8.0430 & 5.8006 & 7.4253 \\
  \hline  
\end{tabular}
\caption{Out-of-sample summary statistics of the optimized strategy
  with transaction cost of 20 basis points,
  and over several different choices of moving average length.}
\label{tab:opt_summary}
\end{table}

The implementation of this optimization is done 
through MATLAB function \emph{fminbnd},
assuming a transaction cost of 20 basis points.
Here we also separate the data into 
in-sample for optimization,
and out-of-sample for testing.
In this experiment, we choose the first half of 
CDS series (1-172),
and the first half of the time period as in-sample,
and the second half of CDS series (173-343)
and the second half of the time period as out-of-sample.
This choice results in no overlap of both 
CDS series and time period between 
the in-sample and out-of-sample data.
The code for this section can be found in \emph{improveStrategy.m}.

Observe the optimization and summary results 
in Table \ref{tab:opt_summary},
we see that even with a transaction cost of 20 basis points,
we can still reach a Sharpe ratio of 2.35 
with an optimized strategy.
Interestingly, we also find the strategy 
improved in terms of kurtosis as well.
Similarly, we can observe the sparsity plot of 
the volume matrix with 3-day moving average.
Even with a choice of small $T$, 
we can have restrict the trading activity to 
less than previously in Figure \ref{fg:spy_volume}.

\begin{figure}[h]
\begin{center}
\epsfig{figure=spy_volume_opt.eps, width=.51\textwidth}
\end{center}
\centering
\caption{\label{fg:spy_volume_opt}
A (sample) sparsity plot of the trading volume matrix
for the optimized strategy with 3-day moving average,
where each point indicates a non-zero entry
in the matrix, i.e. a trade.
}
\end{figure}

In conclusion, we find the optimized strategy 
can perform even under realistic transaction cost
level at 20 basis points.
The strategy at this stage is definitely 
worth pursuing.







\end{document}
